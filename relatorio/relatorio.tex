%%%%%%%%%%%%%%%%%%%%%%%%%%%%% Define Article %%%%%%%%%%%%%%%%%%%%%%%%%%%%%%%%%%
\documentclass[twocolumn]{article}
%%%%%%%%%%%%%%%%%%%%%%%%%%%%%%%%%%%%%%%%%%%%%%%%%%%%%%%%%%%%%%%%%%%%%%%%%%%%%%%

%%%%%%%%%%%%%%%%%%%%%%%%%%%%% Using Packages %%%%%%%%%%%%%%%%%%%%%%%%%%%%%%%%%%
\usepackage{geometry}
\usepackage{graphicx}
\usepackage{amssymb}
\usepackage{amsmath}
\usepackage{amsthm}
\usepackage{empheq}
\usepackage{mdframed}
\usepackage{booktabs}
\usepackage{lipsum}
\usepackage{graphicx}
\usepackage{color}
\usepackage{psfrag}
\usepackage{pgfplots}
\usepackage{bm}
%%%%%%%%%%%%%%%%%%%%%%%%%%%%%%%%%%%%%%%%%%%%%%%%%%%%%%%%%%%%%%%%%%%%%%%%%%%%%%%

\usepackage[brazil]{babel}
\usepackage{babelbib}
\usepackage{url}
\usepackage{titlesec}
\usepackage{hyperref}
\usepackage[acronym,automake]{glossaries-extra}

\titleformat{\section}[runin]{\normalfont\large\bfseries}{}{0pt}{}
\titleformat{\subsection}[runin]{\normalfont\normalsize\bfseries}{}{0pt}{}
\titleformat{\subsubsection}[runin]{\normalfont\small\bfseries}{}{0pt}{}



%%%%%%%%%%%%%%%%%%% Glossary %%%%%%%%%%%%%%%%%%%%%%%%

\makeglossaries
\newacronym{mccrm}{MCCRM}{Minimização de Arrependimento Contrafactual de Monte Carlo}
\newacronym{crm}{CRM}{Minimização de Arrependimento Contrafactual}
\newacronym{ia}{IA}{Inteligência Artificial}

\newglossaryentry{flush}{
    name=flush,
    description={Um jogo em poker em que um jogador possui cinco cartas do mesmo naipe.}
}

\newglossaryentry{arrep}{
    name=arrependimento,
    description={O quanto um agente preferiria ter tomado uma decisão à decisão real dele.}
}

\newglossaryentry{esnash} {
    name={estratégia de Nash},
    description={Uma estratégia ótima em um equillíbrio de Nash. Desvios de uma estratégia de Nash, não podem melhorar a performance
    de um agente.}
}

%%%%%%%%%%%%%%%%%%%%%%%%%% Page Setting %%%%%%%%%%%%%%%%%%%%%%%%%%%%%%%%%%%%%%%
\geometry{a4paper}
\pagenumbering{roman}


%%%%%%%%%%%%%%%%%%%%%%%%%% Define an orangebox command %%%%%%%%%%%%%%%%%%%%%%%%
\newcommand\orangebox[1]{\fcolorbox{ocre}{mygray}{\hspace{1em}#1\hspace{1em}}}
%%%%%%%%%%%%%%%%%%%%%%%%%%%%%%%%%%%%%%%%%%%%%%%%%%%%%%%%%%%%%%%%%%%%%%%%%%%%%%%

%%%%%%%%%%%%%%%%%%%%%%%%%%%% English Environments %%%%%%%%%%%%%%%%%%%%%%%%%%%%%
\newtheoremstyle{mytheoremstyle}{3pt}{3pt}{\normalfont}{0cm}{\rmfamily\bfseries}{}{1em}{{\color{black}\thmname{#1}~\thmnumber{#2}}\thmnote{,--,#3}}
\newtheoremstyle{myproblemstyle}{3pt}{3pt}{\normalfont}{0cm}{\rmfamily\bfseries}{}{1em}{{\color{black}\thmname{#1}~\thmnumber{#2}}\thmnote{,--,#3}}
\theoremstyle{mytheoremstyle}
\newmdtheoremenv[linewidth=1pt,backgroundcolor=shallowGreen,linecolor=deepGreen,leftmargin=0pt,innerleftmargin=20pt,innerrightmargin=20pt,]{theorem}{Theorem}[section]
\theoremstyle{mytheoremstyle}
\newmdtheoremenv[linewidth=1pt,backgroundcolor=shallowBlue,linecolor=deepBlue,leftmargin=0pt,innerleftmargin=20pt,innerrightmargin=20pt,]{definition}{Definition}[section]
\theoremstyle{myproblemstyle}
\newmdtheoremenv[linecolor=black,leftmargin=0pt,innerleftmargin=10pt,innerrightmargin=10pt,]{problem}{Problem}[section]
%%%%%%%%%%%%%%%%%%%%%%%%%%%%%%%%%%%%%%%%%%%%%%%%%%%%%%%%%%%%%%%%%%%%%%%%%%%%%%%

%%%%%%%%%%%%%%%%%%%%%%%%%%%%%%% Plotting Settings %%%%%%%%%%%%%%%%%%%%%%%%%%%%%
\usepgfplotslibrary{colorbrewer}
\pgfplotsset{width=8cm,compat=1.9}
%%%%%%%%%%%%%%%%%%%%%%%%%%%%%%%%%%%%%%%%%%%%%%%%%%%%%%%%%%%%%%%%%%%%%%%%%%%%%%%

%%%%%%%%%%%%%%%%%%%%%%%%%%%%%%% Title & Author %%%%%%%%%%%%%%%%%%%%%%%%%%%%%%%%
\title{Inteligência Artificial de Poker\\Pluribus}
\author{Pedro Tavares de Carvalho}
%%%%%%%%%%%%%%%%%%%%%%%%%%%%%%%%%%%%%%%%%%%%%%%%%%%%%%%%%%%%%%%%%%%%%%%%%%%%%%%

\begin{document}
    \begin{titlepage}
        \maketitle

        \vspace{3cm}
        \abstract{
            Nesse projeto, iremos discutir a inteligência artificial de poker chamada \emph{Pluribus}~\cite{brown2019superhuman},
            seus detalhes de implementação e design, além de seus impactos na sociedade e no futuro de \glsxtrfull{ia} em jogos.
            Abordaremos o poker como um jogo em teoria de jogos, seu processo de modelagem e os desafios
            existentes ao se tratar de um jogo como o poker em uma inteligência artificial.
            }

    \end{titlepage}

    \section{Introdução} % (fold)
    \label{sec:Introdução}
        O poker é um dos jogos mais antigos e mais jogados do mundo. Com regras simples de se entender, porém
        difíceis de se dominar, o poker atende jogadores casuais e aficionados da mesma forma.

        Apesar disso, a revolução das inteligências artificiais, que inclui damas~\cite{5392560}, xadrez~\cite{CAMPBELL200257}
        e mais recentemente GO~\cite{Silver2016}, atingiu diversos outros jogos antes do mesmo. Muito disso é por causa
        da dificuldade de se modelar uma estratégia ótima em poker.

        A história da inteligência artifical no poker é longa, incluindo diversas estratégias, dentre elas algoritmos genéticos~\cite{4219032} e
        aprendizado profundo~\cite{DBLP:journals/corr/MoravcikSBLMBDW17}. Uma versão mais simplificada do poker, onde os jogadores possuem
        limites em quanto eles podem apostar, já foi fechada matematicamente~\cite{bowling2015heads}.

        Iremos explorar no que o Pluribus difere das estratégias empregadas até o momento, e o motivo dessas estratégias serem menos eficientes
        e menos efetivas do que as novas, modelando o poker matematicamente e observando os detalhes que fizeram o Pluribus quebrar a barreira
        do poker sem limites com 6 jogadores.

    % section Introdução (end)

    \section{Modelagem do Poker em Teoria de Jogos } % (fold)
    \label{sec:Modelagem do Poker em Teoria de Jogos }
        O poker pode ser modelado em teoria de jogos como um jogo:


        \begin{description}
            \item[De informação incompleta] Os agentes do jogo não possuem acesso a quais cartas os seus adversários estão na mão, ou seja,
                com as informações que ele tem no momento, não é possível determinar com certeza em qual estado de jogo o mesmo está.

            \item[Adversarial] Os agentes jogam com o objetivo de ganhar, sem colaborar entre si.

            \item[Soma zero] Quando um dos agentes joga, uma combinação dos outros perde, tornando o valor total existente no jogo fixo.

            \item[Estocástico] Ações dentro do poker podem produzir estados diferentes, e, consequentemente, valores diferentes.
        \end{description}

        Em geral, o poker difere de jogos como xadrez e GO por alguns motivos, incluindo o fato de ele ser um jogo de informação incompleta
        e estocástico, e pelo fato de o mesmo ser jogado com mais de dois jogadores, o que torna a estratégia e a busca dentro da árvore de estados
        mais complexa do que um minimax simples.

        \subsection{Equilíbrio de Nash e Poker GTO} % (fold)
        \label{ssub:Equilíbrio de Nash e Poker GTO}
            O poker, quando jogado por pessoas reais, é dividido em duas vertentes principais, a vertente exploitativa, que tenta se aproveitar
            de erros na estratégia do seu adversário, e o poker Ótimo em Teoria dos Jogos (GTO), que tenta maximizar matemáticamente a estratégia,
            tentando atingir um Equilíbrio de Nash~\cite{nash1950equilibrium}.

            Em poker sem limites, não existe uma estratégia de Nash fechada, o que é feito são tentativas de aproximação dessa estratégia,
            no caso do poker humano, com conceitos como valor esperado, Independant Chip Modelling~\cite{gilbert2009independent} e outros conceitos
            que ajudam a aproximar a melhor ação em cada momento. Já em inteligências artificiais, a técnica mais usada é a de \glsxtrfull{crm}, que atualiza a árvore de decisões tentando diminuir a quantidade de decisões erradas.


        % subsubsection Equilíbrio de Nash e Poker GTO (end)
    % section Modelagem do Poker em Teoria de Jogos  (end)

    \section{O Pluribus por alto} % (fold)
    \label{sec:O Pluribus por alto}
        O Pluribus é composto de três peças principais, abstração de informação, o croquê de estratégia construído por minimização de \gls{arrep},
        e o algoritmo de busca em tempo real que cria estratégia em nós não antes explorados.

        A estratégia do Pluribus foi construída, principalmente, por jogos com versões de si mesmo. Essa estratégia é muito utilizada em inteligências
        artificiais, incluindo o xadrez~\cite{DBLP:journals/corr/abs-1712-01815} e GO~\cite{Silver2016}. Com esses jogos, o Pluribus desenvolve uma estratégia
        base, que é utilizada e atualizada no momento da busca em tempo real, quando o agente está realmente jogando.
    % section O Pluribus por alto (end)

    \section{Abstrações de Informação } % (fold)
    \label{sec:Simplificação de Informação }
        Para simplificar a busca e melhorar a estratégia da \glsxtrshort{ia}, o Pluribus aplica técnicas de simplificação de informação, tanto a nível de apostas
        possíveis quanto a nível de valor de mão.

        \subsection{Abstração de Apostas} % (fold)
        \label{sub:Simplificação de Apostas}

            A simplificação de apostas age tanto no espaço de apostas consideradas pelo agente em sua jogada quanto em apostas de adversários. O processo
            é uma discretização das apostas, transformando valores específicos de apostas em um conjunto específico de apostas possíveis.

        % subsection Simplificação de Apostas (end)

        \subsection{Abstração de Mãos} % (fold)
        \label{sub:Simplificação de Mãos}
            Esse processo diminui o espaço de mãos possíveis dentro de um jogo de poker. A idéia é que mão de valores similares, como um \emph{flush}
            com carta alta em 10 e um \emph{\gls{flush}} com carta alta em 9, serão agrupadas em um mesmo conjunto, e ações similares (se não idênticas) serão
            tomadas em situações que surgem com essas mãos.

            Isso não é uma regra, e dependendo da especificidade da situação, a abstração pode ser desconsiderada, e as mãos podem ser tratadas como
            diferentes pela estratégia.

        % subsection Simplificação de Mãos (end)

    % section Simplificação de Informação  (end)

    \section{Minização de arrependimento contrafactual de Monte Carlo} % (fold)
    \label{sec:Minização de arrependimento contrafactual de Monte Carlo}
        Para a construção da estratégia base, é utilizada uma técnica chamada de \glsxtrfull{mccrm}~\cite{NIPS2009_00411460,brown2019solving}.

        Essa técnica consiste na simulação de jogos da \glsxtrshort{ia} contra si mesma, e da revisão das decisões tomadas no jogo completo a fim de minimizar o "\gls{arrep}"
        do agente, ou seja, aumentando a probabilidade de decisões boas e diminuindo a de decisões ruins.

        Diferente da CRM normal, em que a cada geração da estratégia a árvore completa de decisões é atualizada, o \glsxtrshort{mccrm} simula somente um jogo\footnote{Um jogo é uma descida
        completa na árvore de decisões, até se chegar em um nó folha.} de todos os possíveis, e atualiza as decisões que são geradas nesse jogo somente.

    % section Minização de arrependimento contrafactual de Monte Carlo (end)

    \section{Busca em jogos de informação incompleta} % (fold)
    \label{sec:Busca em Jogos de informação incompleta}

        Diferente de jogos de infomação completa, como xadrez e GO, em que você consegue ter certeza do estado em que você está e da estratégia do seu adversário,
        busca em ambientes de informação incompleta não conseguem se basear apenas na avaliação de um estado na \gls{esnash}, ao menos não de um estado não-folha.

        Isso acontece pois a avaliação assume que o adversário não usará uma estratégia diferente no futuro, ou seja, continuará utilizando uma \gls{esnash}
        ótima, e portanto tomará as mesmas decisões. Isso pode ser visto na figura \ref{fig:rps}, onde uma busca em um jogo de pedra papel ou tesoura
        causa uma estratégia que não implica em otimalidade.

        Por exemplo, se tivéssemos um algoritmo guloso, uma busca poderia sempre resultar em \emph{pedra}
        como ação, o que faria com que se um adversário racional mudasse sua estratégia, ele pudesse explorar os nossos vieses.


        \begin{figure}
            \includegraphics[width=0.48\textwidth]{./rps.jpeg}
            \caption{Avaliação e busca em um jogo de pedra papel ou tesoura. Primeiro se vê a descida das decisões e valores possíveis
            na árvore, depois se vê a avaliação de cada decisão em uma \gls{esnash}.}
            \label{fig:rps}
        \end{figure}

        Para contornar esse problema, o Pluribus aplica uma técnica de pesquisa com profundidade limitada~\cite{brown2018depth} em que diferentes estratégias
        adversárias são emuladas e avaliadas, observando o valor em todas elas. Para o Plúribus, foram utilizadas 4 estratégias diferentes, sendo essas
        vieses construídos a partir da \emph{blueprint} de estratégia gerada com o \hyperref[sec:Minização de arrependimento contrafactual de Monte Carlo]{\glsxtrshort{mccrm}}.



    % section Busca em Jogos de informação incompleta (end)


    \clearpage
    \printglossary
    \printglossary[type=\acronymtype]
    \nocite{*}
    \bibliographystyle{bababbrv}
    \bibliography{referencias}

\end{document}
